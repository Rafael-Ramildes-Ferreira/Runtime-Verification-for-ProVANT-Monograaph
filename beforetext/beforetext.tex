% ---
% Capa
% ---
\imprimircapa
% ---

% ---
% Folha de rosto
% (o * indica que haverá a ficha bibliográfica)
% ---
\imprimirfolhaderosto*
% ---

% ---
% Inserir a ficha bibliografica
% ---
% http://ficha.bu.ufsc.br/
\begin{fichacatalografica}
	\includepdf{beforetext/Ficha_Catalografica.pdf}
\end{fichacatalografica}
% ---

% ---
% Inserir folha de aprovação
% ---
\begin{folhadeaprovacao}
	\OnehalfSpacing
	\centering
	\imprimirautor\\%
	\vspace*{10pt}		
	\textbf{\imprimirtitulo}%
	\ifnotempty{\imprimirsubtitulo}{:~\imprimirsubtitulo}\\%
	%		\vspace*{31.5pt}%3\baselineskip
	\vspace*{\baselineskip}
	
	This dissertation was evaluated in the context of the subject DAS5511 (Course Final Project) and approved in its final form by the \imprimircurso\\
	%Esta monografia foi julgada no contexto da disciplina DAS5511 (Projeto de Fim de Curso) e aprovada em sua forma final pelo \imprimircurso\\
	\vspace*{\baselineskip}
	Florianópolis, <month> <day>, <year>.\\
	
	%%%%%%%%%%%%%%%%%%%%%%%%%%%%%%%%%%%%%%%%%%%%%
	%IMPORTANT: no signatures are required below!
	%%%%%%%%%%%%%%%%%%%%%%%%%%%%%%%%%%%%%%%%%%%%%
	
	\vspace*{2\baselineskip}
	%\rule{0.4\textwidth}{0.4pt}\\
	Prof. xxxx, Dr.\\
	Course Coordinator\\
	
	\vspace*{\baselineskip}
	\textbf{Examining Board:} \\
	
	
	\vspace*{2\baselineskip}
	%\rule{0.4\textwidth}{0.4pt}\\
	Prof. xxxx, Dr.\\
	Advisor \\
	UFSC/CTC/EAS\\
	
	\vspace*{2\baselineskip}
	%\rule{0.4\textwidth}{0.4pt}\\
	xxxx, Eng.\\
	Supervisor \\
	Company/University xxxx\\
	
	\vspace*{2\baselineskip}
	%\rule{0.4\textwidth}{0.4pt}\\
	Prof. xxxx, Dr.\\
	Evaluator \\
	Institution xxxx\\
	
	\vspace*{2\baselineskip}
	%\rule{0.4\textwidth}{0.4pt}\\
	Prof. xxxx, Dr.\\
	Board President \\
	UFSC/CTC/EAS
\end{folhadeaprovacao}
% ---

% ---
% Dedicatória
% ---
\begin{dedicatoria}
	\vspace*{\fill}
	\noindent
	\begin{adjustwidth*}{}{5.5cm} 
		\raggedleft       
		This work is dedicated to my classmates and my dear parents.
	\end{adjustwidth*}
\end{dedicatoria}
% ---

% ---
% Agradecimentos
% ---
\begin{agradecimentos}
	Inserir os agradecimentos aos colaboradores à execução do trabalho. 
	
	Xxxxxxxxxxxxxxxxxxxxxxxxxxxxxxxxxxxxxxxxxxxxxxxxxxxxxxxxxxxxxxxxxxxxxx. 
\end{agradecimentos}
% ---

% ---
% Epígrafe
% ---
\begin{epigrafe}
	\vspace*{\fill}
	\begin{flushright}
		\textit{``Texto da Epígrafe.\\
			Citação relativa ao tema do trabalho.\\
			É opcional. A epígrafe pode também aparecer\\
			na abertura de cada seção ou capítulo.\\
			Deve ser elaborada de acordo com a NBR 10520.''\\
			(SOBRENOME do autor da epígrafe, ano)}
	\end{flushright}
\end{epigrafe}
% ---

% ---
% DECLARAÇÃO DE PUBLICIDADE
% ---

\begin{center}
	\textbf{DISCLAIMER}
\end{center}

% Atenção: atualize o conteúdo de <texto>.

<City name>, <month> <day>-th, <year>.

\vspace{1cm}

As representative of the <PFC institution of execution> in which the present work was carried out, I declare this document to be exempt from any confidential or sensitive content regarding intellectual property, that may keep it from being published by the Federal University of Santa Catarina (UFSC) to the general public, including its online availability in the Institutional Repository of the University Library (BU). Furthermore, I attest knowledge of the obligation by the author, as a student of UFSC, to deposit this document in the said Institutional Repository, for being it a Final Program Dissertation (\emph{``Trabalho de Conclusão de Curso''}), in accordance with the \emph{Resolução Normativa n° 126/2019/CUn}.

\vspace{15mm}

\begin{center}
	\rule{7cm}{0.7pt} \\
	<Fulano de Tal> \\
	<Instituição de realização do PFC>
\end{center}

\cleardoublepage

% ---
% RESUMOS
% ---

% resumo em português
\setlength{\absparsep}{18pt} % ajusta o espaçamento dos parágrafos do resumo
\begin{resumo}
	\SingleSpacing
	\textbf{Instruções do padrão genérico de TCCs da BU:}
	No Abstract são ressaltados o objetivo da pesquisa, o método utilizado, as discussões e os resultados com destaque apenas para os pontos principais. O Abstract deve ser significativo, composto de uma sequência de frases concisas, afirmativas, e não de uma enumeração de tópicos. Não deve conter citações. Deve usar o verbo na voz ativa e na terceira pessoa do singular. O texto do Abstract deve ser digitado, em um único bloco, sem espaço de parágrafo. O espaçamento entre linhas é simples e o tamanho da fonte é 12. Abaixo do Abstract, informar as palavras-chave (palavras ou expressões significativas retiradas do texto) ou, termos retirados de thesaurus da área. Deve conter de 150 a 500 palavras. O Abstract é elaborado de acordo com a NBR 6028. 
	
	\textbf{Instruções da Coordenação de PFC:} O Abstract deve descrever de forma sucinta: o contexto/motivação/problema tratado no PFC; a solução proposta; a implementação/desenvolvimento; a metodologia e as principais técnicas e ferramentas utilizadas; os principais resultados obtidos e a importância/impactos de tais resultados para a empresa/clientes da empresa/instituto de pesquisa. Escrever todos esses pontos de forma bem resumida e direta, e sem entrar em detalhes técnicos. O tamanho do Abstract deve ocupar praticamente esta página inteira, e num \textbf{único} parágrafo. Além disso, Abstract + Keywords não podem ultrapassar esta página. 
	
	\textbf{Keywords}: Keyword 1. Keyword 2. Keyword 3. \emph{[essas palavras-chave devem obrigatoriamente ser utilizadas no Abstract]}
\end{resumo}

% resumo em inglês
\begin{resumo}[Resumo]
	\SingleSpacing
	\begin{otherlanguage*}{brazil}
		Resumo traduzido para outros idiomas, neste caso, português. Segue o mesmo formato do Abstract.
		
		\textbf{Palavras-chave}: Palavra-chave 1. Palavra-chave 2. Palavra-chave 3.
	\end{otherlanguage*}
\end{resumo}

%% resumo em francês 
%\begin{resumo}[Résumé]
% \begin{otherlanguage*}{french}
%    Il s'agit d'un résumé en français.
% 
%   \textbf{Mots-clés}: latex. abntex. publication de textes.
% \end{otherlanguage*}
%\end{resumo}
%
%% resumo em espanhol
%\begin{resumo}[Resumen]
% \begin{otherlanguage*}{spanish}
%   Este es el resumen en español.
%  
%   \textbf{Palabras clave}: latex. abntex. publicación de textos.
% \end{otherlanguage*}
%\end{resumo}
%% ---

{%hidelinks
	\hypersetup{hidelinks}
	% ---
	% inserir lista de ilustrações
	% ---
	\pdfbookmark[0]{\listfigurename}{lof}
	\listoffigures*
	\cleardoublepage
	% ---
	
	% ---
	% inserir lista de quadros
	% ---
	\pdfbookmark[0]{\listofquadrosname}{loq}
	\listofquadros*
	\cleardoublepage
	% ---
	
	% ---
	% inserir lista de tabelas
	% ---
	\pdfbookmark[0]{\listtablename}{lot}
	\listoftables*
	\cleardoublepage
	% ---
	
	% ---
	% inserir lista de abreviaturas e siglas (devem ser declarados no preambulo)
	% ---
	\imprimirlistadesiglas
	% ---
	
	% ---
	% inserir lista de símbolos (devem ser declarados no preambulo)
	% ---
	\imprimirlistadesimbolos
	% ---
	
	% ---
	% inserir o sumario
	% ---
	\pdfbookmark[0]{\contentsname}{toc}
	\tableofcontents*
	\cleardoublepage
	
}%hidelinks
% ---