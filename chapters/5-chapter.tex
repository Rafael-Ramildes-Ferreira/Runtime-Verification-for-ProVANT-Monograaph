% ----------------------------------------------------------
\chapter{Tests and results}
% ----------------------------------------------------------

% ----------------------------------------------------------
\section{Simulation configuration} 
% ----------------------------------------------------------
O que tava conectado em quê, como pegamos os dados...

% ----------------------------------------------------------
\section{Results}
% ----------------------------------------------------------

% \textbf{Instruções da Coordenação do PFC:}

% Neste capítulo, deve-se:
% \begin{itemize}
% 	\item Descrever detalhadamente como foi feita a implementação/desenvolvimento da solução proposta;
% 	\item Deixar bem claro e justificar tecnicamente para o leitor como que o desenvolvimento realizado implementa de fato a solução proposta, explicando tecnicamente as decisões que foram tomadas para se chegar a tal implentação;
% 	\item Analisar os resultados obtidos com base em indicadores, gráficos, estatísticas, etc: 
% 	\begin{itemize}
% 		\item A implementação realizada solucionou de fato o problema tratado? 
% 		\item Obteve-se o resultado esperado? 
% 		\item Mostrou-se melhor que o método anterior?
% 		\item Vantagens e desvantagens; 
% 		\item Problemas encontrados;   
% 		\item Impacto dos resultados obtidos nos processos/projetos/produtos/serviços/clientes da empresa/instituto de pesquisa; 
% 		\item Impactos organizacionais, tecnológicos, financeiros, éticos, ecológicos, etc.
% 	\end{itemize}
% \end{itemize}

% Sugere-se colocar uma diagrama/fluxograma ilustrando como que a solução proposta foi implementada/desenvolvida, e depois explicar em detalhes cada parte/bloco do diagrama/fluxograma ao longo do texto. 

% Ressaltamos que, em princípio, existe uma infinidade de maneiras diferentes de implementar a solução proposta. Desse modo, o diagrama/fluxograma da solução proposta apresentado no capítulo anterior é mais geral e abstrato que o diagrama/fluxograma da implementação: a implementação realizada no PFC é uma maneira específica de se chegar à solução proposta a partir das técnicas, ferramentas e métodos utilizados. 

